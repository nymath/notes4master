\section{主成分分析}
主成分分析可以用于降维,同时保持数据的大部分变差。一般来说,当数据相关性较强时,使用主成分分析的效果较好。

\begin{sdefinition}{}{}
主成分应当具有如下性质:
\begin{itemize}
	\item 每一个主成分是原始变量的线性组合
	\item 主成分数目远远小于变量的数量
	\item 主成分保留了原始变量的大部分信息
	\item 第一主成分应当保留最多的信息
	\item 各主成分之间互不相关
\end{itemize}	                                                                    
\end{sdefinition}
\begin{sremark}{}{}
我们这里提到的信息,使用方差进行刻画。
\end{sremark}

\begin{stheorem}{Real Spectrum}{}
实数域上的自伴随正算子(在标准正交基下为正定的实对称矩阵),有如下分解
$$
\tcbhighmath{\M(T,\nu,\nu) = \M(I,e,\nu)\M(T,e,e)\M(I,\nu,e)}
$$
其中$\nu$为特征向量构成的基,而且
$$
\tcbhighmath{\M(T,\nu,\nu) = diag(\lambda_1,\cdots,\lambda_n)}
$$
$$
\tcbhighmath{\M(I,\nu,e)=[\nu_1,\cdots,\nu_n]} 
$$
即$T$在特征基下的矩阵为一个对角矩阵。
\end{stheorem}

\subsection{协方差矩阵法}
\begin{stheorem}{}{}
假定$\Sigma$正定,$\lambda_i$为其特征值,$\nu_i$为对应的特征向量,则
$$
\max_{x \perp \text{span}\{\nu_1,\cdots,\nu_k\}} \frac{\inner{\Sigma x}{x}}{\inner{x}{x}} = \lambda_{k+1}
$$
\end{stheorem}


\begin{stheorem}{}{}
假定p维随机向量的协方差矩阵为$\M(T,e,e)=\Sigma$,则正交变换
$$
\tcbhighmath{Y = \M(T,e,\nu) X=P^{-1} X}.
$$
生成了p个主成分,且$Y[k]=\nu_k^\prime X$为第k主成分。
\end{stheorem}
值得注意的是,
$$
Var(Y_i) = Var(\nu_k \prime X) = \nu_k \prime \Sigma \nu_k = \lambda_k.
$$
\begin{stheorem}{PC的性质}{}
\begin{itemize}
	\item $Y$的协方差矩阵为$\M(T,\nu,\nu)$
	\item $Y$的总方差为$\M(T,\nu,\nu)$的trace.
\end{itemize}
\end{stheorem}
\begin{Proof}
\begin{eqnarray*}
	Var(\iota^\prime Y) &=& tr(\M(T,\nu,\nu)) \\
	&=& tr(\M(T,e,e))
\end{eqnarray*}
\end{Proof}

\begin{sdefinition}{贡献率}{}
主成分$Y_i$的贡献率定义为:
$$
\frac{\lambda_i}{\sum_{i=1}^{p}\lambda_i}
$$
累积贡献率定义为
$$
\frac{\sum_{i=1}{m} \lambda_i}{\sum_{i=1}^{p}\lambda_i}
$$
\end{sdefinition}

\begin{sdefinition}{因子负荷量(Factor Loadings)}{}
$X_i$与$Y_k$的Pearson rho定义为因子负荷量
$$
\tcbhighmath{\rho(Y_k,X_i) = \frac{\nu_{ik} \sqrt{\lambda_k}}{\sqrt{\sigma_{ii}}}}
$$
\end{sdefinition}
\begin{Proof}
	\begin{eqnarray*}
		Cov(Y_k,X_i) &=& Cov(\nu_k^\prime X,e_i^\prime X) \\
		&=& \nu_k^\prime \Sigma e_i \\
		&=& \lambda_k \nu_{ik}.
	\end{eqnarray*}
\end{Proof}
\begin{sremark}{}{}
$\nu_{ik}$可以解释为第k个主成分对第i个变量的敏感性。 \\
而因子负荷量的绝对值大小刻画了该主成分的主要意义及其成因,在解释第i个变量是第k个主成分的重要性时,应该根据因子负荷量进行分析,而不是仅仅$\nu_{ik}$. 
\end{sremark}

\begin{stheorem}{}{}
$$
	\tcbhighmath{\sum_{i=1}^p \rho^2(Y_k,X_i)\sigma_{ii} = \lambda_k}
$$
\end{stheorem}

\subsection{相关系数法}


\subsection{主成分的几何含义}
随机向量$X$在$e$下的坐标为$(X_1,\cdots,X_p)$,这些指标的coordinate不是独立的。然后呢,我们找了一个新的坐标系$\nu = P\circ e$,在这个坐标系下,$X$的坐标为$(Y_1,\cdots,Y_p)$,而且这些坐标之间是独立的。同时呢,我们也稍微的对$P$的行(列)进行一些交换。这么做之后达到了一个怎么样的效果呢?$X$在$\nu_1$方向上的变差,也就是$Var(Y_1)$,最大。(这体现为,对X不断抽样,然后把它投影到$\nu_1$上,轨迹应该最长才对)










