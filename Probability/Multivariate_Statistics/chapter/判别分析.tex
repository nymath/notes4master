\section{判别分析}


\subsection{贝叶斯判别法}
这里顺便复习一下,贝叶斯统计。假设总体有k个类别,我们用$\Theta$表示,$\Theta \sim (p_1,\cdots,p_k)$. 用$X$代表数据展现出的特征,同时我们有条件密度函数
$$
f(x|\Theta=i) = f_i(x)
$$
这个条件密度函数可以通过样本进行估计(或者直接给定,但前者更好)。
基于此,我们得到参数的后验分布,$\Theta_{|X=x}$,他服从
$$
\Theta_{|X=x} \sim (\frac{f_i(x)P_i}{\sum_{i=1}^kP_if_i(x)})
$$
接下来,我们想选择一个随机变量(估计量),他与$\Theta_{|X=x}$的"距离"充分接近。
常用的备选有
\begin{itemize}
	\item $L^2$距离: $L(\theta,\Theta_{|X=x})=E(\theta-\Theta_{|X=x})^2$.
	\item $L^1$距离: $L(\theta,\Theta_{|X=x})=E(|\theta-\Theta_{|X=x}|)$.
	\item 
\end{itemize}
\begin{sremark}{}{}
特别的,在$L^2$距离意义下,我们的估计是$E(\Theta|X=x)$.
\end{sremark}
但是呢,在判别分析中,我们有更一般的损失函数,即
$$
L(\theta,\Theta_{|X=x})= E(c(\theta|\Theta_{|X=x})).
$$
\begin{sexample}{}{}
当$c(\theta|\Theta_{|X=x})=1-\chi_{\theta}(\Theta_{|X=x})$时,期望损失为
$$
1 - \sum_{i=1}^{n}\chi_{\theta}(i)\frac{f_i(x)P_i}{\sum_{i=1}^kP_if_i(x)}
$$
因此,当$\theta$等于$\Theta_{|X=x}$的众数时,期望损失最小。
\end{sexample}
更一般的,我们的损失函数可以定义为
$$
\tcbhighmath{L(\theta,\Theta_{|X=x}) = \sum_{j=1}^k c(\theta|j)\frac{f_j(x)P_j}{\sum_{i=1}^kP_if_i(x)}} 
$$

\subsection{Fisher判别法}
Fisher判别法的思路在于,$\R^p$的一个向量,将我们的各组数据投影在这个向量上,使得组内距离充分小,组间距离充分大。(实际上是一维的方差分析)。由于我们的数据都在同一个方向,所以我们只需要计算各组数据与这个向量的内积即可。具体而言
$$
\tcbhighmath{
\begin{array}{l l l l l}
	G_1: & x_{1}^{(1)}, &x_{2}^{(1)}, & \cdots, &  x_{n_1}^{(1)}  \\
	G_2: & x_{1}^{(2)}, &x_{2}^{(2)}, & \cdots, &  x_{n_2}^{(2)}  \\
	 &  & \vdots & &  \\
	G_k: & x_{1}^{(k)}, &x_{2}^{(k)} ,& \cdots, &  x_{n_k}^{(k)}  \\
\end{array}
}
$$

