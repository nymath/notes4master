\documentclass[12pt]{article}
\usepackage{ctex}

\usepackage[top=.8in, bottom=.8in, left=.8in, right=.8in]{geometry}
\usepackage{amsmath}
\usepackage{tikz}
\usetikzlibrary[topaths]
\newcount\mycount
\usepackage{amssymb,latexsym}
\usepackage{amsxtra}
\usepackage{amsthm}
\usepackage{graphicx}
\usepackage{setspace}
\onehalfspacing
\usepackage{wasysym} 
\usepackage{verbatim} 
\usepackage{arcs}
\usepackage{accents}
\theoremstyle{definition}
\usepackage[skins]{tcolorbox}
\tcbuselibrary{skins,theorems}
\tcbuselibrary{breakable}
%---------------------- New-environment ----------------
\newenvironment{Proof}
{
\vspace{-0.5cm}
\paragraph{\textcolor[rgb]{0.1,0.8,0.9}{\textsf{Proof}}}
}
{\hfill$\square$\vspace{0.3cm}}

\newenvironment{Remark}
{
\paragraph{\textcolor[rgb]{0.9,0.0,0.6}{\textsf{Remark}}}
}

%----------------------

\usepackage{hyperref}
\hypersetup{
    colorlinks=true,
    linkcolor=black,
    filecolor=magenta,      
    urlcolor=cyan,
    pdftitle={Overleaf Example},
    pdfpagemode=FullScreen,
    }
\urlstyle{same}


%-----------------------------------------------------------------------------%
\newtheorem{definition}{Definition}[section]
\newtheorem{example}{Example}[section]
\newtheorem{theorem}{Theorem}[section]
\newtheorem{proposition}{Proposition}[section]
\newtheorem*{fact}{Fact}
\newtheorem{remark}{Remark}
\newtheorem{corollary}{Corollary}[section]
\newtheorem{lemma}{Lemma}[section]


\newcommand{\df}{\displaystyle \frac} 
\newcommand{\dlim}{\displaystyle \lim}
\newcommand{\dint}{\displaystyle \int}
\newcommand{\ra}{\rangle}
\newcommand{\la}{\langle}
\newcommand{\inner}[2]{{\langle #1,#2\rangle}}
\newcommand{\x}{\mathbf{x}}
\newcommand{\xt}{\mathbf{x}^{\mathsf{T}}}
\newcommand{\T}{{\mathsf{T}}}
\newcommand{\abf}{\mathbf{a}}
\newcommand{\abft}{\mathbf{a}^{\mathsf{T}}}
\newcommand{\R}{\mathbb{R}}
\newcommand{\C}{\mathbb{C}}
\newcommand{\E}{\mathrm{e}}
\newcommand{\F}{\mathbb{F}}
\newcommand{\X}{\mathbf{X}}
\newcommand{\Y}{\mathbf{Y}}
\newcommand{\f}{\mathbf{f}}
\newcommand{\U}{\mathbf{u}}
\newcommand{\D}{\mathrm{d}}
\newcommand{\MCG}{\mathcal{G}}
\newcommand{\MCF}{\mathcal{F}}
\newcommand{\M}{\mathcal{M}}
\newcommand{\MCB}{\mathcal{B}}
\newcommand{\MCT}{\mathcal{T}}
\newcommand{\LL}{\mathcal{L}}
\newcommand{\nullspace}{\mathrm{null}}
\newcommand{\range}{\mathrm{range}}
\newcommand{\Sum}[2]{{\sum_{#1}^{#2}}}
\newcommand{\Union}[2]{{\bigcup_{#1}^{#2}}}
\newcommand{\Intersection}[2]{{\bigcap_{#1}^{#2}}}


\newcommand{\pd}[1]{\frac{\partial}{\partial #1}}
\definecolor{my-violet}{rgb}{0.9,0.0,0.6}

\newtcbtheorem[auto counter,number within=section]{stheorem}{定理}
{
	breakable,
	fonttitle=\bfseries\upshape,description font=\bfseries\itshape,
	arc=3mm, ,separator sign = \ ,
	theorem number and name,
	colbacktitle=blue!25!white,colback=blue!8!white,colframe=black,coltitle=black,
	toptitle=1.5mm,bottomtitle=1.5mm, 
}{th}

\newtcbtheorem[use counter from=stheorem]{sdefinition}%
  {定义}{breakable,fonttitle=\bfseries\upshape,description font=\bfseries\itshape,
     arc=3mm, ,separator sign = \ ,
     theorem number and name,
     colbacktitle=yellow!50!white,colback=yellow!15!white,colframe=black,coltitle=black,
     toptitle=1.5mm,bottomtitle=1.5mm,
     }{def}
     
\newtcbtheorem[use counter from=stheorem]{sconclude}%
  {}{breakable,fonttitle=\bfseries\upshape,description font=\bfseries\itshape,
     arc=3mm, ,separator sign = \ ,
     theorem number and name,
     colbacktitle=yellow!50!white,colback=yellow!15!white,colframe=black,coltitle=black,
     toptitle=1.5mm,bottomtitle=1.5mm,
     }{def}

\newtcbtheorem[use counter from=stheorem]{scorollary}%
  {定理}{breakable,fonttitle=\bfseries\upshape,description font=\bfseries\itshape,
     arc=3mm, ,separator sign = \ ,
     theorem number and name,
     colbacktitle=blue!25!white,colback=blue!8!white,colframe=black,coltitle=black,
     toptitle=1.5mm,bottomtitle=1.5mm,
     }{cr}
     
\newtcbtheorem[use counter from=stheorem]{sexample}%
  {Example}{breakable,theorem style=plain,separator sign = \ ,
  	fonttitle=\sffamily\normalsize,description font=\sf\itshape,
  	colbacktitle=black!11!white,coltitle=black,colframe=black,
    arc=0mm,outer arc=0mm,
    toptitle=1.5mm,bottomtitle=1.5mm,left=1mm,right=1mm,
	boxrule=0mm,toprule=0.5mm,bottomrule=.5mm,	rightrule=0mm,titlerule=0mm,top=0mm,
	,colback=black!6!white}{ex}

\newtcbtheorem[]{sremark}%
  {Remark}{breakable,theorem style=plain,theorem name,
  	description font=\sf\itshape,fontupper=\sffamily\normalsize, colbacktitle=green!11!white,coltitle=my-violet,colframe=black,colback=black!6!white,
    arc=2mm,
    left=1mm,right=1mm,
	toprule=0.5mm,bottomrule=.5mm,rightrule=0.5mm,leftrule=.5mm,titlerule=0mm,
	}{ex}
\newtcbox{\boxred}{colback=red!5!white,
colframe=red!75!black,on line,size=fbox}
\newtcbox{\boxgrey}{on line,size=fbox}
\newtcbox{\Ebox}{colback=red!5!white,
colframe=red!75!black}


%\usepackage[backend=biber,style=gb7714-2015]{biblatex}
%\setlength{\bibitemsep}{3bp}
%\renewcommand*{\bibfont}{\zihao{5}\linespread{1.27}\selectfont}
%\addbibresource{bibliography.bib}


%--------------------------------------------

\begin{document}
	\title{An Preliminary Introduction to Copula Theory}
	\author{Ny 
	\footnote{我会不定期更新笔记,如果感兴趣的话,可以前往
	\href{https://github.com/nymath/notes4master}{https://github.com/nymath/notes4master}获取最新版本。}\\
	nymath@163.com}
	\date{\today}
	\maketitle
	\abstract{Copula的一些简要介绍\cite{czado2019analyzing} }
	\tableofcontents
	\listoffigures
\newpage
\section{导论}
设想我们想要生成一个具有分布函数$F$的随机变量$X$,应该如何做呢?
\begin{stheorem}{随机数生成}{}
如果$U\sim U(0,1)$,即U是一个均匀分布。$F$为随机变量$X$的分布函数(单增且右连续), 如果F的存在反函数
$$
F^{-1}: [0,1] \mapsto \R
$$
则
$F^{-1}(U)$与$X$同分布。

\end{stheorem}
\begin{sremark}{}{}
$X$是一个随机变量,$F$是他的cdf,则$F \circ X = F(X)$服从$Uniform(0,1)$。	
\end{sremark}
上述定理告诉我们,想要得到具有分布函数$F$的随机变量的一个样本,我们只需要先模拟一个均匀分布的样本,然后带入$F^{-1}$计算即可。\par

\boxred{随机变量容易模拟,那随机向量呢}?模拟随机向量不仅需要模拟边缘分布,还需要模拟相关性结构。 从定理1.1我们发现,模拟一个随机变量只需要关注这个均匀随机变量$U$即可。类似的,如果我们想模拟随机向量,我们得知道边缘分布函数,以及这些随机变量的\boxred{依赖性}。这使得我们把工作重心转移到了刻画均匀随机向量$(U_1,\cdots,U_p)$的相关性结构上。而\boxred{Copula},正是用于研究这种相关性结构。 \par
Copula英文含义为连系动词,目前没有中文翻译(可以把它叫做链接函数,但和link function有点冲突),
与量化投资中常常提到的alpha类似,$Copula$在不同的场景下有不同的含义,目前我遇到的主要有两种。




\begin{sdefinition}{Copula定义1}{}
Copula是一个累积分布函数,他定义在$[0,1]^p$上,即
$$
C: [0,1]^p \to [0,1]
$$
\end{sdefinition}

\begin{sexample}{常见Copula}{}
\begin{enumerate}
	\item 独立Copula: $C(u_1,\cdots,u_d) = \Pi_{k=1}^p u_k$
	\item 共单调(Comonotonicity)Copula: $C(u_1,\cdots,u_d) = \min\{u_1,\cdots,u_d\}$
	\item 反单调Copula():
	\item Gaussian Copula: 
\end{enumerate}
\end{sexample}


\begin{sdefinition}{Copula定义2}{}
Copula还是一个随机向量$(U_1,\cdots,U_p)$,这个随机向量的联合分布函数是我们上边定义的那个$C$,即满足
$$
\mathrm{Pr}(U_1\leq u_1,\cdots,U_p \leq u_p) = C(u_1,\cdots,u_p)
$$
\end{sdefinition}
\begin{stheorem}{两种定义的等价性}{}
一方面,给定均匀向量$(U_1,\cdots,U_p)$,通过下式可以定义一个函数$C$
$$
C(u_1,\cdots,u_p) = \mathrm{Pr}(U_1\leq u_1,\cdots,U_p \leq u_p) = C(u_1,\cdots,u_p)
$$
另一方面,如果给定一个联合累积分布函数$C$,我们也知道了均匀随机向量的相关性信息。
\end{stheorem}
\begin{sremark}{}{}
上述两个定义是等价的,以后我们提到Copula这个词,需要根据具体场景判断是一个联合累积分布函数,还是一个均匀随机向量。
\end{sremark}

\section{Copula的一些重要结论}
\subsection{一般随机向量和Copula的关系}
可能读到这里会稍有疑惑,我们想模拟一般的随机向量$(X_1,\cdots,X_p)$, 但现在怎么去模拟均匀向量$(U_1,\cdots,U_p)$了呢?这里我用一个不太严谨的图来表示这种关系
\begin{stheorem}{一些简要解释}{}
我们认为知道\boxred{联合分布}函数的信息后,就能知道随机向量$X$的信息,反之亦然。可以认为
$$
\text{联合分布函数的信息} = \text{各个边缘分布的信息}
 + \text{变量间的相依性结构}
$$
而Copula正是刻画这种相关性信息。因此只要给定了边缘分布后,我们就能模拟出一般的随机向量$X$。不严谨的,我们可以认为
$$
\text{联合分布函数} = \text{各个边缘分布函数} +  \text{Copula}
$$
\end{stheorem}

\begin{stheorem}{Sklar's Theorem}{}
给定Copula(均匀向量),$(U_1,\cdots,U_p)$,以及边缘分布函数$F_i$,则随机向量可以表示为
$$
(X_1,\cdots,X_p) = (F_1^{-1}(U_1),\cdots,F_p^{-1}(U_p))
$$
而且$(X_1,\cdots,X_p)$的联合累积分布函数$F$等于
$$
F(x_1,\cdots,x_p) = C(F_1(x_1),\cdots,F_p(x_p))
$$
\end{stheorem}
上述定理用一个严格一些的论述,便是著名的Sklar's Theorem,他是Copula理论的核心,请多加思考这个定理。
先给出一个通俗的叫法,
\begin{sdefinition}{一些通俗叫法}{}
我们称随机向量$(X,Y)$具有copula $C$,如果
$$
(F_1(X),F_2(Y)) \text{具有累积分布函数C}
$$
\end{sdefinition}
\begin{stheorem}{Sklar's Theorem}{}
如果随机向量$X$具有联合cdf,$F$以及边缘分布函数$F_i$,则$X$具有Copula $C$ which is define as follows:
$$
C(u_1,\cdots,u_p) = F(F_1^{-1}(u_1),\cdots,F_p^{-1}(u_p))。
$$
\end{stheorem}







\begin{stheorem}{Copula的单调不变性}{}
假如$f,g$是单调递增函数(事实上单调映射即可), 且$(X,Y)$具有copula C, 则
$$
f(X), g(Y) \text{也具有Copula C}.
$$
\end{stheorem}
\begin{sremark}{}{}
特别的,当这个单调函数取作随机变量$X,Y$的累积分布函数时,我们得到
$$
(U_1,U_2) = (F_X(X),F_Y(Y))
$$
也具有copula C.
\end{sremark}
上述几个定理告诉我们,只要把均匀随机向量$U$的结构弄清楚了,要模拟出一般的随机向量$X$,自然是手到擒来。
所以从现在开始,所有的讨论全部集中于均匀随机向量了。



\section{Dependence Measures}
这里的Measure并不是测度论中的测度,就是一个度量指标罢了,对于相依性的度量,我们从相关系数,尾部相依指数展开。
\subsection{相关系数}
\subsubsection{Pearson rho}
\begin{sdefinition}{总体相关系数}{}
Suppose $X,Y$ are random variables on a probability space $(\Omega,\mathcal{F},P)$, then the Pearson Rho coefficient of $X,Y$ is defined by 
$$
\rho_p(X,Y) = \frac{E((X-E(X)(Y-E(Y))}{\sqrt{Var(X)}\sqrt{Var(Y)}} = \frac{\inner{E-E(X)}{Y-E(Y)}}{\Vert X-E(X)\Vert_2 \Vert Y-E(Y) \Vert_2}
$$
\end{sdefinition}
\begin{sremark}{}{}
通过Pearson rho的定义,我们不难发现,随机变量$X,Y$的相关系数其实是离差变量X-E(X)与Y-E(Y)之间的夹角。(另外值得注意的是,随机变量X的方差其实就是在$L^2$范数诱导的距离意义下,X到均值E(X)距离的平方)
\end{sremark}

\begin{sdefinition}{样本相关系数}{}
Suppose $X_i$ and $Y_i$ are random samples of $X$ and $Y$, then the sample Pearson rho 
coefficient is defined by
$$
\hat \rho_p(X,Y) = \frac{\Sum{i=1}{n}(X_i -\bar X)(Y_i -\bar Y)}{\sqrt{\Sum{i=1}{n}(X_i-\bar X)^2}\sqrt{\Sum{i=1}{n}(X_i-\bar X)^2}}
$$
\end{sdefinition}
\begin{sremark}{}{}
	相当于我们用样本对总体方差和总体协方差进行了一个估计,这个估计应该是一致的。
\end{sremark}

\subsubsection{Spearman rho}
\begin{sdefinition}{总体相关系数}{}
假如随机变量$X,Y$的边缘分布函数是$F_X, F_Y$,那么$X,Y$之间的总体Spearman rho相关系数定义为$X,Y$诱导的Copula的Pearson rho相关系数,即$F_X(X), F_Y(Y)$的Pearson rho相关系数。
$$
\rho_s(X,Y) = \rho_p (F_X(X),F_Y(Y))
$$
\end{sdefinition}
\begin{sremark}{}{}
	通过总体Spearman rho相关系数的定义我们不难发现,由于单调变换不改变(X,Y)的Copula,即f(X),g(Y)与X,Y有相同的Copula,自然f(X), g(Y)的spearman rho系数不变。这说明了Spearman rho度量是随机变量生成的copula的相关性,并不Care他们的边缘分布,也就是说它的确可以度量一些非线性关系。
\end{sremark}

\begin{sdefinition}{样本相关系数}{}
类似的,我们可以利用样本对X,Y的spearman rho进行估计,
$$
\hat{\rho}_s\left(X, Y\right):=\frac{\sum_{i=1}^n\left(r_{i 1}-\bar{r}_1\right)\left(r_{i 2}-\bar{r}_2\right)}{\sqrt{\sum_{i=1}^n\left(r_{i 1}-\bar{r}_1\right)^2} \sqrt{\sum_{i=1}^n\left(r_{i 2}-\bar{r}_2\right)^2}},
$$
其中$r_{i1}$代表$X_i$在样本中的rank。
\end{sdefinition}
\begin{sremark}{}{}
如何理解这个公式是总体spearman rho的一个估计?我们可以分子分母同时除以样本容量$n$的平方,把
$\frac{r_{i1}}{n}$看作均匀变量$U_1$的一次观测,自然上式就成为了$U_1,U_2$的Pearson rho的一个估计。	
\end{sremark}



\subsubsection{Kendall tau}
\begin{sdefinition}{总体相关系数-kendall-tau}{}
The Kendall's $\tau$ between the continuous random variables $X_1$ and $X_2$ is defined as
$$
\tau\left(X_1, X_2\right)=P\left(\left(X_{11}-X_{21}\right)\left(X_{12}-X_{22}\right)>0\right)-P\left(\left(X_{11}-X_{21}\right)\left(X_{12}-X_{22}\right)<0\right),
$$
where $\left(X_{11}, X_{12}\right)$ and $\left(X_{21}, X_{22}\right)$ are independent and identically distributed copies of $\left(X_1, X_2\right)$.
\end{sdefinition}
kendall tau的系数估计有点复杂,在实际中,由于部分Copula的参数和Kendall相关系数有一个函数,所有我们
可以通过估计Kendall tau系数进而得到Copula参数的估计值。

\subsection{各个相关系数之间的关系}
\begin{stheorem}{关联}{}
假设$X,Y$服从二维正态分布,则
$$
\rho_p=2 \sin \left(\frac{\pi}{6} \rho_s\right) \text { and } \tau=\frac{2}{\pi} \arcsin (\rho)
$$
\end{stheorem}

$$
\begin{array}{l|l|l}
\hline \text { Family } & \text { Kendall's } \tau & \text { Range of } \tau \\
\hline \text { Gaussian } & \tau=\frac{2}{\pi} \arcsin (\rho) & {[-1,1]} \\
\hline \mathrm{t} & \tau=\frac{2}{\pi} \arcsin (\rho) & {[-1,1]} \\
\hline \text { Gumbel } & \tau=1-\frac{1}{\delta} & {[0,1]} \\
\hline \text { Clayton } & \tau=\frac{\delta}{\delta+2} & {[0,1]} \\
\hline \text { Frank } & \tau=1-\frac{4}{\delta}+4 \frac{D_1(\delta)}{\delta} \text { with } & {[-1,1]} \\
& D_1(\delta)=\int_0^\delta \frac{x / \delta}{e^x-1} d x \text { (Debye function) } &
\end{array}
$$

\subsection{尾部相依性}
\begin{sdefinition}{上尾相依系数}{}
The upper tail dependence coefficient of a bivariate distribution with copula $C$ is defined as
$$
\lambda_U=\lim _{t \rightarrow 1^{-}} P\left(X_2>F_2^{-1}(t) \mid X_1>F_1^{-1}(t)\right)=\lim _{t \rightarrow 1^{-}} \frac{1-2 t+C(t, t)}{1-t},
$$
\end{sdefinition}

\begin{sdefinition}{下尾相依系数}{}
while the lower tail dependence coefficient is
$$
\lambda_L=\lim _{t \rightarrow 0^{+}} P\left(X_2 \leq F_2^{-1}(t) \mid X_1 \leq F_1^{-1}(t)\right)=\lim _{t \rightarrow 0^{+}} \frac{C(t, t)}{t} .
$$
\end{sdefinition}
$$
\begin{array}{l|l|l}
\hline \text { Family } & \text { Upper tail dependence } & \text { Lower tail dependence } \\
\hline \text { Gaussian } & - & - \\
\hline \mathrm{t} & 2 t_{\nu+1}\left(-\sqrt{\nu+1} \sqrt{\frac{1-\rho}{1+\rho}}\right) & 2 t_{\nu+1}\left(-\sqrt{\nu+1} \sqrt{\frac{1-\rho}{1+\rho}}\right) \\
\hline \text { Gumbel } & 2-2^{1 / \delta} & - \\
\hline \text { Clayton } & - & 2^{-1 / \delta} \\
\hline \text { Frank } & - & - \\
\hline \text { Joe } & 2-2^{1 / \delta} & - \\
\hline \text { BB1 } & 2-2^{1 / \delta} & 2^{-1 /(\delta \theta)} \\
\hline \text { BB7 } & 2-2^{1 / \theta} & 2^{-1 / \delta} \\
\hline \text { Galambos } & 2^{-1 / \delta} & - \\
\hline \text { BB5 } & 2-\left(2-2^{-1 / \delta}\right)^{1 / \theta} & - \\
\hline \text { Tawn } & \left(\psi_1+\psi_2\right)-\left(\psi_1^\theta+\psi_2^\theta\right)^{1 / \theta} & - \\
\hline \text { t-EV } & 2\left[1-T_{\nu+1}\left(z_{1 / 2}\right)\right] & - \\
\hline \text { Hüsler-Reiss } & 2\left[1-\Phi\left(\frac{1}{\lambda}\right)\right] & - \\
\hline \text { Marshall-Olkin } & \min \left\{\alpha_1, \alpha_2\right\} & - \\
\hline
\end{array}
$$
\section{Bivariate Copula Classes}
本节主要讨论两变量的copula,当然部分结论也适用于多变量的情形。
\subsection{Gaussian Copula}
本文介绍正态Copula的模拟,首先给出正态Copula的定义,
\begin{sdefinition}{正态copula}{}
假设$\Phi$具有协方差矩阵$\Sigma$(对角线为1)的$p$维正态分布的联合累积分布函数,$\varphi$则是标准正态分布的累积分布函数,则正态Copula($\Sigma$)定义为
$$
C(u_1,\cdots,u_p) = \Phi(\varphi^{-1}(u_1),\cdots,\varphi^{-1}(u_p))
$$
\end{sdefinition}

接下来是正态Copula的模拟,事实上,我们只需要先模拟一个具有协方差矩阵$\Sigma$的多元正态向量,$(Z_1,\cdots,Z_p)$,然后利用
$$
(U_1,\cdots,U_p) = (\varphi(Z_1),\cdots,\varphi(Z_p)).
$$
即可得到具有协方差矩阵$\Sigma$的正态Copula,这里有一个代码可以参考一下,见附件。
这里选用了000651.SZ和601318.SH	的简单收益率进行分析,我们绘制了散点图,左图是正态Copula,边缘分布也为正态(实际上就是多元正态分布),右图是正态Copula,边缘分布为Gamma分布,可以发现,利用正态Copula+Gamma分布能够更好的拟合极端情形。
以然后我用正态Copula模拟

\begin{figure}[htb]
	\centering
	\includegraphics[scale=0.4]{figure/Copula_contrast.png}
	\caption{正态Copula下不同边缘分布的对比}
\end{figure}

\subsection{t Copula}
\section{Archimedean Copulas}
\begin{sdefinition}{Archimedean Copula}{}
Suppose $\Omega = \{\varphi:[0,1] \to [0,\infty) | \text{$\varphi$ is a continuous,
strictly monotone decreasing,} \\ \text{, and convex function.} \}$ Let $\varphi \in \Omega$, then 
$$
C(u_1,\cdots,u_p) = \varphi^{[-1]}(\varphi(u_1)+\cdots+\varphi(u_p))
$$
is indeed a copula, where  $\varphi^{[-1]}$ is $\varphi$'s pseudo-inverse defined by
$$
\varphi^{[-1]}(t) = \varphi^{-1}(t) \chi_{[0,\varphi(0)]}(t)
$$
\end{sdefinition}


 


%\bibliographystyle{siam}
%\bibliography{bibliography.bib}
%\printbibliography[heading = bibintoc]


\end{document}