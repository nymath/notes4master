\usepackage[top=.8in, bottom=.8in, left=.8in, right=.8in]{geometry}
\usepackage{amsmath}
\usepackage{ctex}
\usepackage{tikz}
\usetikzlibrary[topaths]
\newcount\mycount
\usepackage{amssymb,latexsym}
\usepackage{amsxtra}
\usepackage{amsthm}
\usepackage{xcolor,colortbl}
\usepackage{graphicx}
\usepackage{setspace}
\onehalfspacing
\usepackage{wasysym} 
\usepackage{verbatim} 
\usepackage{arcs}
\usepackage{accents}
\theoremstyle{definition}
\usepackage[skins]{tcolorbox}
\usepackage{changepage}

\tcbuselibrary{skins,theorems}
\tcbuselibrary{breakable}
%---------------------- New-environment ----------------
\newenvironment{Proof}
{
\vspace{-0.5cm}
\paragraph{\textcolor[rgb]{0.1,0.8,0.9}{\textsf{Proof}}}
}
{\hfill$\square$\vspace{0.3cm}}

\newenvironment{Remark}
{
\paragraph{\textcolor[rgb]{0.9,0.0,0.6}{\textsf{Remark}}}
}

%----------------------

\usepackage{hyperref}
\hypersetup{
    colorlinks=true,
    linkcolor=black,
    filecolor=magenta,      
    urlcolor=cyan,
    pdftitle={Overleaf Example},
    pdfpagemode=FullScreen,
    }
\urlstyle{same}




\setlength{\arrayrulewidth}{0.5mm} %This sets the thickness of the borders of the table
%\setlength{\tabcolsep}{18pt}
\renewcommand{\arraystretch}{1.6} 

%-----------------------------------------------------------------------------%
\newtheorem{definition}{Definition}[section]
\newtheorem{example}{Example}[section]
\newtheorem{theorem}{Theorem}[section]
\newtheorem{proposition}{Proposition}[section]
\newtheorem*{fact}{Fact}
\newtheorem{remark}{Remark}
\newtheorem{corollary}{Corollary}[section]
\newtheorem{lemma}{Lemma}[section]

\newcommand{\bfw}[1]{\textbf{\textcolor{white}{#1}}}

\newcommand{\df}{\displaystyle \frac} 
\newcommand{\dlim}{\displaystyle \lim}
\newcommand{\dint}{\displaystyle \int}
\newcommand{\ra}{\rangle}
\newcommand{\la}{\langle}
\newcommand{\inner}[2]{{\langle #1,#2\rangle}}
\newcommand{\x}{\mathbf{x}}
\newcommand{\xt}{\mathbf{x}^{\mathsf{T}}}
\newcommand{\T}{{\mathsf{T}}}
\newcommand{\abf}{\mathbf{a}}
\newcommand{\abft}{\mathbf{a}^{\mathsf{T}}}
\newcommand{\R}{\mathbb{R}}
\newcommand{\C}{\mathbb{C}}
\newcommand{\E}{\mathrm{e}}
\newcommand{\F}{\mathbb{F}}
\newcommand{\X}{\mathbf{X}}
\newcommand{\Y}{\mathbf{Y}}
\newcommand{\f}{\mathbf{f}}
\newcommand{\U}{\mathbf{u}}
\newcommand{\D}{\mathrm{d}}
\newcommand{\MCG}{\mathcal{G}}
\newcommand{\MCF}{\mathcal{F}}
\newcommand{\M}{\mathcal{M}}
\newcommand{\MCB}{\mathcal{B}}
\newcommand{\MCT}{\mathcal{T}}
\newcommand{\LL}{\mathcal{L}}
\newcommand{\nullspace}{\mathrm{null}}
\newcommand{\range}{\mathrm{range}}
\newcommand{\Sum}[2]{{\sum_{#1}^{#2}}}
\newcommand{\Union}[2]{{\bigcup_{#1}^{#2}}}
\newcommand{\Intersection}[2]{{\bigcap_{#1}^{#2}}}

\newcommand{\pd}[1]{\frac{\partial}{\partial #1}}
\definecolor{my-violet}{rgb}{0.9,0.0,0.6}
\definecolor{my-darkred}{RGB}{138, 38, 38}
\definecolor{my-red}{RGB}{209, 177, 161}
\definecolor{myblue1}{RGB}{183, 200, 230}
\definecolor{myblue0}{RGB}{225, 231, 245}
\definecolor{myyellow0}{RGB}{254, 246, 167}
\definecolor{myyellow1}{RGB}{254, 250, 204}


\newtcbtheorem[auto counter,number within=chapter]{stheorem}{Theorem}
{
	breakable,
	fonttitle=\bfseries\upshape,description font=\bfseries\itshape,
	arc=3mm, ,separator sign = \ ,
	theorem number and name,
	colbacktitle=myblue1,colback=myblue0,colframe=black,coltitle=black,
	toptitle=1.5mm,bottomtitle=1.5mm, 
}{thm}

\newtcbtheorem[use counter from=stheorem]{sdefinition}%
  {Definition}{breakable,fonttitle=\bfseries\upshape,description font=\bfseries\itshape,
     arc=3mm, ,separator sign = \ ,
     theorem number and name,
     colbacktitle=myyellow0,colback=myyellow1,colframe=black,coltitle=black,
     toptitle=1.5mm,bottomtitle=1.5mm,
     }{def}
     
\newtcbtheorem[use counter from=stheorem]{sconclude}%
  {}{breakable,fonttitle=\bfseries\upshape,description font=\bfseries\itshape,
     arc=3mm, ,separator sign = \ ,
     theorem number and name,
     colbacktitle=yellow!50!white,colback=yellow!15!white,colframe=black,coltitle=black,
     toptitle=1.5mm,bottomtitle=1.5mm,
     }{def}

\newtcbtheorem[use counter from=stheorem]{scorollary}%
  {Theorem}{breakable,fonttitle=\bfseries\upshape,description font=\bfseries\itshape,
     arc=3mm, ,separator sign = \ ,
     theorem number and name,
     colbacktitle=myblue1,colback=myblue0,colframe=black,coltitle=black,
     toptitle=1.5mm,bottomtitle=1.5mm,
     }{cr}
     
\newtcbtheorem[use counter from=stheorem]{sexample}%
  {Example}{breakable,theorem style=plain,separator sign = \ ,
  	fonttitle=\sffamily\normalsize,description font=\sf\itshape,
  	colbacktitle=black!11!white,coltitle=black,colframe=black,
    arc=0mm,outer arc=0mm,
    toptitle=1.5mm,bottomtitle=1.5mm,left=1mm,right=1mm,
	boxrule=0mm,toprule=0.5mm,bottomrule=.5mm,	rightrule=0mm,titlerule=0mm,top=0mm,
	,colback=black!6!white}{ex}
\newtcbtheorem[use counter from=stheorem]{algorithm}%
  {Algorithm}{breakable,theorem style=plain,separator sign = \ ,
  	fonttitle=\sffamily\normalsize,description font=\sf\itshape,
  	colbacktitle=black!11!white,coltitle=black,colframe=black,
    arc=0mm,outer arc=0mm,
    toptitle=1.5mm,bottomtitle=1.5mm,left=1mm,right=1mm,
	boxrule=0mm,toprule=0.5mm,bottomrule=.5mm,	rightrule=0mm,titlerule=0mm,top=0mm,
	,colback=black!6!white}{ex}

\newtcbtheorem[]{sremark}%
  {Remark}{breakable,theorem style=plain,theorem name,
  	description font=\sf\itshape,fontupper=\sffamily\normalsize, colbacktitle=green!11!white,coltitle=my-violet,colframe=black,colback=black!6!white,
    arc=2mm,
    left=1mm,right=1mm,
	toprule=0.5mm,bottomrule=.5mm,rightrule=0.5mm,leftrule=.5mm,titlerule=0mm,
	}{ex}
	
	
\newtcbox{\boxred}{colback=red!5!white,
colframe=red!75!black,on line,size=fbox}
\newtcbox{\boxgrey}{on line,size=fbox}
\newtcbox{\Ebox}{colback=red!5!white,
colframe=red!75!black}
